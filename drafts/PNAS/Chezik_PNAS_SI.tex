\documentclass[9pt]{pnas-new}

\usepackage{upgreek}

\templatetype{pnasresearcharticle}

\title{River networks dampen long-term hydrological signals of climate change}

% Use letters for affiliations, numbers to show equal authorship (if applicable) and to indicate the corresponding author
\author[a,1]{Kyle A. Chezik}
\author[b,1]{Sean C. Anderson} 
\author[a,1]{Jonathan W. Moore}

\affil[a]{Earth to Ocean Research Group, Department of Biological Sciences, Simon Fraser University, 8888 University Dr., Burnaby, British Columbia V5A 1S6, Canada}
\affil[b]{School of Aquatic and Fishery Sciencies, University of Washington, Box 455020, Seattle, WA 98195, USA}

% Please give the surname of the lead author for the running footer
\leadauthor{Chezik} 

\begin{document}

\maketitle

\section*{Supplementary Information}
The following supplementary material includes 5 additional figures. Figures \ref{fig:S1} and \ref{fig:S2} are similar to figure 4 in the main text where monthly trends in flow are shown for the basin along with trend attenuation results. In these supplementary figures we highlight minimum and median flow trends rather than maximum flow trends. Figure \ref{fig:S3} demonstrates the effect of the seasonal transition of spring and fall on the climate portfolio index values within and among Fraser River sub-basins. Figures \ref{fig:S4} and \ref{fig:S5} are example plots demonstrating a single null-model iteration for the median flow metric.

\subsection*{Supplementary Information Figures}

\begin{figure*}[b]
\centering
\includegraphics[width=17.8cm]{FigS1_MinMonth.pdf}
	\caption{Monthly minimum flow trend attenuation within the Fraser River basin. (\textbf{Left}) Fraser River's basin-wide minimum-flow trend estimates (i.e., intercept = vertical grey lines) by month with density distributions of null-model simulations. Observed values falling further from the center of the density distribution suggest greater evidence for changes in minimum flow and a greater shift in magnitude. (\textbf{Center}) Observed monthly Fraser River minimum-flow variance exponent ($\hat{\updelta}$, blue) and associated density distribution of simulated $\hat{\updelta}$ estimates. Decimal values represent the percent of simulated data exhibiting weaker attenuation (yellow) than observed. (\textbf{Right}) Trend estimates $\pm$ one standard error (SE, grey) plotted against watershed area (km\textsuperscript{2}), colored by climate portfolio strength (green = small, blue = large), for four seasonally representative months. These reflect months in the prior columns and describe the variation in percent change per decade of minimum flow among sites. Simulated lines ignore variance in the intercept and slope to focus visually on attenuation.}
\label{fig:S1}
\end{figure*}

\begin{figure*}
\centering
\includegraphics[width=17.8cm]{FigS2_MedMonth.pdf}
	\caption{Monthly median flow trend attenuation within the Fraser River basin. (\textbf{Left}) Fraser River's basin-wide median-flow trend estimates (i.e., intercept = vertical grey lines) by month with density distributions of null-model simulations. Observed values falling further from the center of the density distribution suggest greater evidence for changes in median flow and a greater shift in magnitude. (\textbf{Center}) Observed monthly Fraser River median-flow variance exponent ($\hat{\updelta}$, blue) and associated density distribution of simulated $\hat{\updelta}$ estimates. Decimal values represent the percent of simulated data exhibiting weaker attenuation (yellow) than observed. (\textbf{Right}) Trend estimates $\pm$ one standard error (SE, grey) plotted against watershed area (km\textsuperscript{2}), colored by climate portfolio strength (green = small, blue = large),for four seasonally representative months. These reflect months in the prior columns and describe the variation in percent change per decade of median flow among sites. Simulated lines ignore variance in the intercept and slope to focus visually on attenuation.}
\label{fig:S2}
\end{figure*}

\begin{figure*}
\centering
\includegraphics[width=11.4cm]{FigS3_SeasonClimPort.pdf}
	\caption{Climate portfolio index by month at 55 sites within  the Fraser River basin. Lines are loess smoothers with a smoothing span of 1 to reduce noise and highlight general season shifts in climate portfolio among and across Fraser River sub-basins.}
	\label{fig:S3}
\end{figure*}

\begin{figure*}
	\includegraphics[width=17.8cm]{FigS4_SiteSim.pdf}
	\caption{Example site specific flow simulations. Annual median-flow simulations (yellow) and observations (blue) for all 55 sites considered in this study found within the Fraser River basin. Simulations at each site were parameterized using the observed data but no trend was imposed. Fifty-five site simulations equate to one basin-wide simulation. Each flow metric was simulated 1000 times basin-wide per response variable. See supplemental figure \ref{fig:S5} for resultant basin-wide simulation.}
	\label{fig:S4}
\end{figure*}

\begin{figure*}
\centering
\includegraphics[width=11.4cm]{FigS5_BasinSim.pdf}
	\caption{Example flow simulation results. Trend estimates $\pm$SE of simulated null-model data (yellow) and the observed data upon which simulations were based (blue), plotted against watershed area. Curved lines represent the estimated variance function ($\hat{\updelta}$) given the simulated data. This is a single example result of the 1000 null-model simulations produced from the data in supplemental figure \ref{fig:S3}.}
	\label{fig:S5}
\end{figure*}

\end{document}
